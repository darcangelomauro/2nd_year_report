\subsection{Yang-Mills matrix models from Dirac operators}
Yang-Mills matrix models have been studied extensively in the past both analytically XXXX(some citations) and numerically XXXX(other citations, like denjoe). Fuzzy spaces arise as classical solutions of these models, and perturbations around them give rise to non-commutative gauge theories XXXX(harold review).\newline
The purpose of this section is to show a formal similarity between Yang-Mills matrix models and a certain choice of Dirac operator, despite the very different origin of the two models.\newline

\subsubsection{Yang-Mills matrix models}
A Yang-Mills matrix model is given by the following action:
\begin{equation}\label{eq:ymaction}
S \propto -\frac{1}{4} \Tr [M_i, M_j]^2 + \frac{2}{3} i g_3 \epsilon_{ijk} \Tr M_i M_j M_k + \frac{1}{2} g_2 \Tr M_i^2
\end{equation}
where $M_i, i=1,2,3$, are $n \times n$ Hermitian matrices and a sum is intended on repeated indices. Actions without the quadratic or cubic term can also be considered. Note that in general one can consider models with more than three matrices (for example the IKKT model XXXX(ikkt) has 10 matrices).\newline
The variation of this action gives the following classical equation of motion for the generic matrix $M_i$:
\begin{equation}\label{eq:ymeom}
[M_j, [M_j, M_i]] + i g_3 \epsilon_{ijk}[M_j, M_k] + g_2 M_i = 0
\end{equation}
where again sum on repeated indices is implied. Diagonal matrices satisfy Eq.(\ref{eq:ymeom}), but a more interesting class of solutions is given by matrices forming an $n$-dimensional representation of the $su(2)$ Lie algebra:
\begin{equation}\label{eq:ymsol}
[M_i, M_j] = i \epsilon_{ijk} M_k.
\end{equation}
In this sense fuzzy spheres are classical solutions of (\ref{eq:ymaction}).

\subsubsection{Type $(0,3)$ geometry as a Yang-Mills matrix model}
Consider the following Dirac operator:
\begin{equation}
D = \sum_i \sigma_i \otimes M_i
\end{equation}
where $\sigma_i, \ i=1,2,3$ are the Pauli matrices and $M_i$ are Hermitian matrices. Note that no commutators or anti-commutators appear in the Dirac operator, meaning that the bimodule structure of the algebra on itself is ignored and only the left (or right) action is taken into account. The Dirac operator is nevertheless an example of type $(0,3)$ geometry.\newline
Consider then the usual action for $D$ (with an additional cubic term and some convenient numerical factors):
\begin{equation}
S[D] = \frac{1}{4} \Tr D^4 + \frac{1}{3}g_3 \Tr D^3 + \frac{1}{4}g_2 \Tr D^2.
\end{equation}
It follows from the properties of the Pauli matrices that such action, rewritten explicitly in terms of the $M_i$ matrices, takes a form very similar to (\ref{eq:ymaction}):
\begin{align}
S[D(M_i)] = &-\frac{1}{4} \Tr [M_i, M_j]^2 -\frac{1}{2} \Tr M_i^2 M_j^2 \\ \notag
&+ \frac{2}{3} i g_3 \epsilon_{ijk} \Tr M_i M_j M_k \\ \notag
&+ \frac{1}{2} g_2 \Tr M_i^2.
\end{align}
The only difference is in fact an extra term $\Tr M_i^2 M_j^2$ in the quartic part. The role of this term becomes clear by writing the equation of motion:
\begin{equation}
[M_j, [M_j, M_i]] - \{M_j^2, M_i \} + i g_3 \epsilon_{ijk}[M_j, M_k] + g_2 M_i = 0
\end{equation}
which still has the $su(2)$ algebra solution (\ref{eq:ymsol}), provided that $\sum_j M_j^2$ is a Casimir. The extra term in the action for the Dirac operator therefore forces the representation to be irreducible.\newline
It is interesting to see what happens if the commutator structure is reintroduced, so that the Dirac operator is now:
\begin{equation}
D = \sum_i \sigma_i \otimes [M_i,\cdot ]
\end{equation}
which is formally equivalent to replacing $M_i$ with $[M_i, \cdot ]$ in each formula so far. First notice that if the Lie algebra relations hold for the matrices, then they hold for the commutators as well:
\begin{equation}
[M_i, M_j] = i \epsilon_{ijk} M_k \ \implies \ [[M_i, \cdot ], [M_j, \cdot ]] = [[M_i, M_j], \cdot ] = i \epsilon_{ijk} [M_k, \cdot ]
\end{equation}
however, it is not clear the role of the extra quartic term in the action, which cannot be interpreted as imposing irreducibility of the classical solution anymore.




\section*{Appendix A: Notation}
\begin{enumerate}
\item To avoid dealing with both Hermitian and anti-Hermitian matrices in Eq.(\ref{eq:dirac_bad}), it is convenient to redefine $i \tilde{L}_j \equiv L_j$ and $\tilde{\alpha}_j \equiv i \alpha_j$. Eq.(\ref{eq:dirac_bad}) becomes:
\begin{equation}
D = \sum_j \tilde{\alpha}_j \otimes [\tilde{L}_j, \cdot ] + \sum_k \tau_k \otimes \{H_k, \cdot \}
\end{equation}
and all the matrices are Hermitian.
\item Commutators and anti-commutators are represented in matrix form as:
$$[A, \cdot ] = A \otimes I - I \otimes A^T$$
$$\{A, \cdot \} = A \otimes I + I \otimes A^T.$$
As a shorthand, the following notation will be used:
\begin{equation}\label{eq:acomm}
[A, \cdot ]_{\epsilon} \equiv A \otimes I + \epsilon \ I \otimes A^T.
\end{equation}
\end{enumerate}
The final form of Eq.(\ref{eq:dirac_bad}) is then:
\begin{equation}\label{eq:dirac}
D = \sum_{i \in I} \omega_i \otimes [ M_i, \cdot ]_{\epsilon_i}
\end{equation}
for Hermitian matrices $M_i$, with $\omega_i \in \{ \tilde{\alpha}_j \} \cup \{ \tau_k \}$, and $\epsilon_i = \pm 1$ depending on $\omega_i$ being a $\tau$ matrix or a $\tilde{\alpha}$ matrix.

\section*{Appendix B: Matrix derivatives} \label{sec:matder}
Let $A \in M_n(\mathbb{C})$ and $f(A)$ be a complex valued function of $A$. The derivative of $f$ with respect to $A$ is defined in components as the $n \times n$ matrix:
\begin{equation}
\left(\frac{\partial f}{\partial A}\right)_{lm} \equiv \frac{\partial f}{\partial A_{lm}}.
\end{equation}
The two special cases of interest here are:
\begin{equation}
\frac{\partial \Tr A}{\partial A} = I
\end{equation}
\begin{equation}
\frac{\partial \Tr AB}{\partial A} = B^T.
\end{equation}

\section*{Appendix C}
The explicit form of $\mathcal{B}_k(i,i,k), \ \mathcal{B}_k(i,k,i)$ and $\mathcal{B}_k(k,k,k)$ is given.
\begin{align}
\mathcal{B}_k(i,i,k) &= \Tr(\omega_k \omega_i \omega_i \omega_k)A(k, i, i, k)^T = CA(k,i,i,k)^T \\ \notag
\mathcal{B}_k(i,k,i) &= \Tr(\omega_k \omega_i \omega_k \omega_i)A(k, i, k, i)^T \\ \notag
\mathcal{B}_k(k,k,k) &= \Tr(\omega_k \omega_k \omega_k \omega_k)A(k, k, k, k)^T = CA(k,k,k,k)^T
\end{align}
where $C$ is the dimension of the Clifford module and the $A$ matrices are:
\begin{align}
A(k,i,i,k) = \ &n [1+\dagger] M_i^2 M_k + \notag \\
&2\epsilon_k I \Tr M_i^2 M_k + \notag \\
&2\epsilon_i \Tr M_i [1 + \dagger] M_i M_k + \notag \\
&4\epsilon_k \epsilon_i M_i \Tr M_i M_k + \notag \\
&2\epsilon_k M_i^2 \Tr M_k + \notag \\
&2M_k \Tr M_i^2
\end{align}
\begin{align}
A(k,i,k,i) = \ &2n M_i M_k M_i + \notag \\
&2\epsilon_k I \Tr M_i^2 M_k + \notag \\
&2\epsilon_i \Tr M_i [1+\dagger] M_i M_k + \notag \\
&4\epsilon_k \epsilon_i M_i \Tr M_i M_k + \notag \\
&2\epsilon_k M_i^2 \Tr M_k + \notag \\
&2M_k \Tr M_i^2
\end{align}
\begin{align}
A(k,k,k,k) = \ &2n M_k^3 + 2\epsilon_k I \Tr M_k^3 + \notag \\
&6 M_k \Tr M_k^2 + 6\epsilon_k M_k^2 \Tr M_k.
\end{align}

\section*{Appendix D}
Suppose $D$ is a $n \times n$ matrix in which $m$ entries are linearly independent. An arbitrary entry can then be written as:
\begin{equation}
D_{ab} = \sum_{ij} c^{ij}_{ab} D_{ij}
\end{equation}
where $c^{ij}_{ab}$ are coefficients and the sum runs over the independent components. This is the case for fuzzy Dirac operators.\newline
The following identity will be proven:
\begin{equation}
\sum_{ij} D_{ij} \frac{\partial}{\partial D_{ij}} \Tr D^p = p \Tr D^p.
\end{equation}
An explicit calculation gives:
\begin{align}
&\sum_{ij} D_{ij} \frac{\partial}{\partial D_{ij}} \Tr D^p = \sum_{ij} D_{ij} \frac{\partial}{\partial D_{ij}} \sum_{a_1 \cdots a_p} D_{a_1 a_2} \cdots D_{a_p a_1} = \\ \notag
&\sum_{ij} D_{ij} \sum_{a_1 \cdots a_p} \left( c^{ij}_{a_1 a_2} D_{a_2 a_3} \cdots D_{a_p a_1} + \cdots + D_{a_1 a_2} \cdots D_{a_{p-1} a_p} c^{ij}_{a_p a_1} \right) = \\ \notag
&\sum_{a_1 \cdots a_p} \left( \left(\sum_{ij} c^{ij}_{a_1 a_2} D_{ij}\right) D_{a_2 a_3} \cdots D_{a_p a_1} + \cdots + D_{a_1 a_2} \cdots D_{a_{p-1} a_p} \left( \sum_{ij} c^{ij}_{a_p a_1} D_{ij} \right) \right) = \\ \notag
&p \sum_{a_1 \cdots a_p} D_{a_1 a_2} \cdots D_{a_p a_1} = p \Tr D^p 
\end{align}
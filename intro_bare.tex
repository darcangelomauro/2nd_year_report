This project is concerned with the characterization of \emph{random fuzzy spaces} by means of Markov chain Monte Carlo simulations. The chapter presents a brief introduction to the basic concepts (non-commutative, fuzzy and random).

\subsection{Non-commutative geometry} \label{sec:ncg}
The fundamental object of non-commutative geometry is a \emph{spectral triple} $(A, H, D)$ where $A$ is an algebra with a representation in a Hilbert space $H$ and $D$ is an operator on $H$, called Dirac operator. A Riemannian spin manifold can be fully characterized by the commutative algebra $A$ of functions on the manifold and by the Dirac operator, which encodes the metric \cite{connesRECON} \cite{connesGRAVMAT}. One could then consider a generalization in which the algebra is allowed to be non-commutative. Such geometries arise naturally in physics, and are tightly related to gauge theories \cite{suijlekom}. Indeed it has been shown that the Standard Model has the structure of a non-commutative geometry \cite{connesSAP} \cite{barrettLOR} \cite{connesNEUT}. This suggests a possible path to quantum gravity by replacing the ordinary commutative spacetime by a non-commutative one that presents a commutative behaviour as a limiting case.

\subsection{Fuzzy spaces} \label{sec:fuzzy}
There is a class of non-commutative geometries called \emph{fuzzy spaces}, where the algebra is taken to be $M_n(\mathbb{C})$, the algebra of $n \times n$ complex matrices, and the Hilbert space is finite dimensional. The Dirac operator of a fuzzy space takes the form \cite{barrett}:
\begin{equation}\label{eq:dirac_bad}
D = \sum_j \alpha_j \otimes [L_j, \cdot ] + \sum_k \tau_k \otimes \{H_k, \cdot \}
\end{equation}
where:
\begin{enumerate}
\item  $\tau_k$ and $\alpha_j$ are respectively Hermitian and anti-Hermitian basis elements of the algebra generated by a $(p, q)$ Clifford module;
\item $H_k$ and $L_j$ are $n \times n$ Hermitian and anti-Hermitian matrices respectively;
\item $[\cdot , \cdot]$ indicates a commutator and $\{ \cdot , \cdot \}$ an anti-commutator.  
\end{enumerate}
Fuzzy spaces are classified by the pair of integers $(p, q)$ of the Clifford module. It is worth noting that any choice of $H_k$ and $L_j$ gives an admissible Dirac operator, as long as the Hermitian or anti-Hermitian character is preserved.\newline
See Appendix A for some remarks on the notation used.

\subsection{Random geometries} \label{sec:rand}
A \emph{random geometry} is a spectral triple $(A, H, D)$ in which the Dirac operator fluctuates according to a certain probability measure. Here the probability measure is taken to be proportional to:
\begin{equation} \label{eq:prob}
e^{-S[D]} d D
\end{equation}
for a certain choice of $S[D]$. The expectation value of an observable $f(D)$ on a random geometry is given by:
\begin{equation} \label{eq:expval}
\langle f(D) \rangle = \int f(D) e^{-S[D]} d D.
\end{equation}
Since $D$ encodes the metric, this is in clear analogy with the Euclidean path integral of Quantum Field Theory. \newline
So far no assumption has been made on the choice of Dirac operator. The purpose of this project is to study the path integral when $D$ is taken to be the Dirac operator of a fuzzy space.\newline
Fuzzy spaces provide an alternative type of regularization that is non-lattice \cite{barrettglaser}. Therefore the study of such random geometries is especially interesting in connection with models of (Euclidean) quantum gravity. \newline
This line of research first appeared in \cite{barrettglaser}, where the following action was considered:
\begin{equation}\label{eq:action}
S[D] = g_2 \Tr D^2 + \Tr D^4, \quad g_2 \in \mathbb{R}.
\end{equation}
In the remainder, the action is taken to be of the form of Eq.(\ref{eq:action}).\newline
The expectation value (\ref{eq:expval}) is computed numerically using Monte Carlo methods. Eq.(\ref{eq:expval}) is therefore replaced by:
\begin{equation} \label{eq:expval_mc}
\langle f(D) \rangle \approx  \frac{1}{N}\sum_{i=1}^N f(D_i)
\end{equation}
where $\{D_i\}$ is a set of Dirac operators sampled from the distribution (\ref{eq:prob}).


